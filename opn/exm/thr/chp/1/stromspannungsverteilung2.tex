\begin{question}[section=1,name={Strom und Spannungsverteilung},difficulty=,quantity=1,type=thr,tags={}]
	Wie hängt die in Dezibel ausgedrückte Dämpfung eines Wellenleiters mit seiner Länge zusammen? Welche Dämpfung hat ein unter optimalen Bedingungen eingesetztes, 100km langes Stück Glasfaserleitung
	\\ \textbf{Hinweis:}\\
	
\end{question}
\begin{solution}
	Die doppelte Länge bedeutet die doppelte Dämpfung in $dB$ gemessen \\
	$ 0,2~dB\cdot100~km=20~dB $
	Wobei angenommen wird das die typische Dämpfung eines Glasfaserwellenleiters bei  $ 0,2 ~dB$ liegt.
\end{solution}