\begin{question}[section=11,name={Rayleightdistanz},difficulty=,quantity=3,type=thr,tags={20151210,20130314}]
	Mit Hilfe welcher Größe (Name) unterscheidet man Nah- und Fernzone einer Antenne und welchen Wert hat sie (Formel)? Geben Sie Bedeutung und Einheit der verwendeten Größen an. 
	\\ \textbf{Hinweis:}\\
	Skript Seite 101
\end{question}
\begin{solution}
	Rayleighdistanz:
	\begin{align}
		r_R &= \frac{2D^2}{\lambda} (+\lambda)\\
	r_R > d &\rightarrow Nahfeld\\
	r_R < d &\rightarrow Fernfeld
\end{align}
	$D$\dots\ maximale Antennenquerabmessung [\metre]\\
	$\lambda$\dots\ Wellenlnge [\metre]\\
	$d$\dots\ Abstand zum Sender [\metre]
\end{solution}