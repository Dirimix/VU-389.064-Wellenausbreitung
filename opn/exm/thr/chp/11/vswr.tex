\begin{question}[section=11,name={Stehwellenverhältnis VSWR},difficulty=,quantity=,type=thr,tags={20151210}]
	Was gibt das Stehwellenverhältnis VSWR an, und wo wird es verwendet?
	\\ \textbf{Hinweis:}\\
	Skript Seite 127
\end{question}
\begin{solution}
	Das Stehwellenverhältnis (VSWR Voltage standing wave ratio) ist ein Gütemerkmal der Anpassung einer Antenne an die Speiseleitung.
	\begin{align}
		m &= VSWR = \frac{1 + |\rho|}{1 - |\rho|} = \frac{|U_{max}|}{|U_{min}|}\\
		\rho &= \frac{Z_G -Z_A}{Z_G + Z_A}
	\end{align}
\end{solution}