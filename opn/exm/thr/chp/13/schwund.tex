\begin{question}[section=13,name={Schwund},difficulty=,quantity=,type=thr,tags={}]
	Wann ist ein System bezüglich des Schwundes schmalbandig und wann breitbandig?
	
	%\\ \textbf{Hinweis:}\\
	
\end{question}
\begin{solution}
	Breitbandiges System: Wenn mehrere Schwundlöcher innerhalb des Systembandbreite sind. Verschiedene Frequenzbereiche schwinden unabhängig voneinander, der Schwund ist frequenzselektiv. $B_S \cdot \Delta \tau_{max} >> 1$\\
	Schmalbandiges System: Wenn das Übertragungsband als Ganzes schwindet. Außerdem, wenn die Bandbreite des Systems wesentlich kleiner als der Frequenzabstand der Schwundlöcher ist. $B_S \cdot \Delta \tau_{max} << 1$
\end{solution}