\begin{question}[section=2,name={Ladungsfreiheit und Relaxationszeit},difficulty=,quantity=5,type=thr,tags={20131210}]
	Was bedeutet der Begriff "`effektive Ladungsfreiheit"'? Durch welche Formel wird die dielektrische Relaxationszeit $\tau_\mathrm{D}$ angegeben und wie gross ist diese näherungsweise bei Kupfer?
	\\ \textbf{Hinweis:}\\
	Skript Seite 14
\end{question}
\begin{solution}
	Effektive Ladungsfreiheit gilt unter der Vorraussetzung, dass große Zeiten gegnüber $\tau_D$ betrachtet werden. Die dielektrische Relaxationszeit für Kupfer ist:
	\begin{equation}
		\tau_\mathrm{D} = \frac{\varepsilon}{\sigma} \approx  10^{-19}\,\second
	\end{equation}
\end{solution}