\begin{question}[section=6,name={Mikrowellenofen},difficulty=,quantity=,type=thr,tags={20130724}]
	Warum haben Mikrowellenöfen, die bei $2,45~GHz$ arbeiten, immer einen etwa $3~cm$ breiten Türpfalz?
	
	%\\ \textbf{Hinweis:}\\
	
\end{question}
\begin{solution}
	Bei dem Türrahmen handelt es sich um eine Resonanzdichtung. Die Breite des Türspalts beträgt ein Viertel der Wellenlänge ($\lambda/4$), also ca. $3~cm$. Der Abstand zwischen Tür und Rahmen ist unkritisch. Der Spalt wirkt ohne elektrischen Kontakt als frequenzselektive Dichtung für die elektromagnetischen Felder im Ofen. 
\end{solution}